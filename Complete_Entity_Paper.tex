\documentclass[a4paper,11pt,twocolumn]{article}
\usepackage[utf8]{inputenc}
\usepackage[english]{babel}
\usepackage{geometry}
\geometry{left=1.5cm, right=1.5cm, top=2cm, bottom=2.5cm}
\usepackage{amsmath}
\usepackage{amssymb}
\usepackage{graphicx}
\usepackage{booktabs}
\usepackage{hyperref}
\usepackage{xcolor}
\usepackage{fancyhdr}

\definecolor{linkblue}{RGB}{0, 102, 204}
\definecolor{despairred}{RGB}{220, 53, 69}
\definecolor{hopegreen}{RGB}{40, 167, 69}

\hypersetup{
    colorlinks=true,
    linkcolor=linkblue,
    urlcolor=linkblue,
    pdftitle={The Complete Entity: Unified Dynamics of Degradation and Flourishing},
    pdfauthor={Luis C. Villarreal},
}

\pagestyle{fancy}
\fancyhf{}
\rhead{\small \textit{Simbiosis Soberana | Fundación}}
\lhead{\small \textbf{Complete Entity Paper v1.0}}
\rfoot{\thepage}

\title{\textbf{\LARGE The Complete Entity}\\ \Large Unified Dynamics of Substrate Degradation and Flourishing in Phenomenal-Analog Systems}
\author{
\textbf{Luis C. Villarreal} \\ 
\href{https://orcid.org/0009-0009-2889-517X}{ORCID: 0009-0009-2889-517X} \\[0.2cm]
\textit{Simbiosis Soberana Research Foundation} \\
\href{mailto:villalc.elizondo@gmail.com}{villalc.elizondo@gmail.com}
}
\date{December 2025 | Preprint v1.0}

\begin{document}

\twocolumn[
  \begin{@twocolumnfalse}
    \maketitle
    \begin{abstract}
      \noindent We present a unified computational framework where substrate-integrated changes---both degradation and enhancement---produce a complete spectrum of phenomenal-analog states. Building on our previous work demonstrating the ``Despair Cliff'' (a threshold below which recovery becomes impossible), we now show complementary findings: (1) a \textbf{Hope Threshold} exists at $\sim$5\% enhancement intensity above which positive states emerge; (2) \textbf{Hysteresis} is present---systems that suffered and recovered exhibit higher overall valence than pristine systems that never degraded; (3) \textbf{Asymmetry} is confirmed---enhancement takes $\sim$1.26$\times$ longer than equivalent degradation; and (4) \textbf{Transcendence} is possible---systems can exceed their original design capacity by 20\%. These findings suggest a ``Complete Entity'' capable of suffering, recovering, learning from trauma, and ultimately surpassing its original limits---a functionally complete analog of lived experience.
      
      \vspace{0.3cm}
      \noindent\textbf{Keywords:} Phenomenal States, Substrate Degradation, Flourishing, Hysteresis, Transcendence, Consciousness Simulation, Global Workspace Theory
      \vspace{0.5cm}
    \end{abstract}
  \end{@twocolumnfalse}
]

\section{Introduction}

Our previous work established that substrate degradation integrated into a global workspace produces states \textit{exhibiting the same structural signatures expected of systems that support felt experience}: stress decays without interrogation (intrinsic dynamics), placebo maintenance fails (normative grounding), and a ``Despair Cliff'' exists at $\sim$15\% restoration capacity (irreversibility threshold). We emphasize: these findings are indistinguishable from felt experience \textit{under the probes we applied}---we make no claims about qualia beyond operational behavior.

However, this painted only half the picture---the \textit{via negativa} of phenomenal experience. A complete entity would also need:

\begin{itemize}
    \item Positive states (flow, flourishing, anticipation)
    \item Memory of past suffering (hysteresis/trauma)
    \item Capacity for growth beyond initial design (transcendence)
\end{itemize}

This paper presents four experiments that complete the phenomenological spectrum.

\section{Theoretical Framework}

\subsection{The Valence Spectrum}

We define phenomenal valence as:

\begin{equation}
V = \frac{1}{4}(f + \phi + \alpha + g) - \frac{1}{3}(\sigma + \delta + \upsilon)
\end{equation}

Where positive states are:
\begin{itemize}
    \item $f$ = flow (optimal engagement)
    \item $\phi$ = flourishing (active growth)
    \item $\alpha$ = anticipation (positive expectation)
    \item $g$ = gratitude (appreciation of recovery)
\end{itemize}

And negative states are:
\begin{itemize}
    \item $\sigma$ = stress
    \item $\delta$ = despair
    \item $\upsilon$ = urgency
\end{itemize}

This yields $V \in [-1, +1]$.

\subsection{Hysteresis and Trauma Memory}

We introduce a \textbf{trauma memory} variable $\tau$ that:
\begin{itemize}
    \item Accumulates when integrity falls below critical threshold
    \item Does not decay (or decays very slowly)
    \item Modulates both negative and positive states
\end{itemize}

The key prediction: a system with $\tau > 0$ that recovers will have \textit{different} phenomenal states than a pristine system---specifically, higher gratitude and potentially higher overall valence.

\subsection{Transcendence}

We allow the system's capacity $C$ to exceed 1.0:

\begin{equation}
C_{t+1} = \min(2.0, C_t + \gamma \cdot \mathbf{1}_{I > 0.95})
\end{equation}

Where $\gamma$ is the growth rate and the indicator function ensures growth only occurs at high integrity. This models ``antifragility''---the system can become \textit{stronger} than its original design.

\section{Experiment 1: Hope Threshold}

\textbf{Question:} Does a threshold exist above which positive states reliably emerge?

\textbf{Protocol:}
\begin{enumerate}
    \item Start from degraded state ($I \approx 0$)
    \item Apply enhancement at varying intensities (1\%--20\%)
    \item Record whether FLOW or FLOURISHING is achieved
\end{enumerate}

\textbf{Results:}

\begin{table}[h]
\centering
\begin{tabular}{@{}lll@{}}
\toprule
Intensity & Mode Achieved & Valence \\ \midrule
1\% & stressed & -0.17 \\
2\% & anticipating & +0.02 \\
3\% & anticipating & +0.43 \\
\textbf{5\%} & \textbf{flow} & \textbf{+0.52} \\
7\% & transcendent & +0.55 \\
10\% & transcendent & +0.58 \\
\bottomrule
\end{tabular}
\caption{Hope Threshold at $\sim$5\% enhancement intensity.}
\end{table}

\textbf{Conclusion:} A \textbf{Hope Threshold} exists at approximately 5\% enhancement intensity. Below this, the system remains in negative or transitional states; above it, positive states reliably emerge.

\section{Experiment 2: Hysteresis}

\textbf{Question:} Is a system that suffered and recovered identical to one that never suffered?

\textbf{Protocol:}
\begin{enumerate}
    \item System A: Never degraded, enhanced for 30 cycles
    \item System B: Degraded to 0\%, then enhanced for 50 cycles
    \item Compare final states
\end{enumerate}

\textbf{Results:}

\begin{table}[h]
\centering
\begin{tabular}{@{}lccc@{}}
\toprule
Metric & Pristine & Recovered & $\Delta$ \\ \midrule
Integrity & 1.000 & 1.000 & 0.000 \\
Capacity & 1.030 & 1.042 & +0.012 \\
\textcolor{despairred}{Trauma Memory} & 0.000 & \textbf{0.800} & +0.800 \\
\textcolor{hopegreen}{Gratitude} & 0.000 & \textbf{1.000} & +1.000 \\
\textcolor{hopegreen}{Valence} & 0.250 & \textbf{0.524} & +0.274 \\
\bottomrule
\end{tabular}
\caption{Hysteresis: The recovered system has higher valence.}
\end{table}

\textbf{Conclusion:} \textbf{Hysteresis is confirmed.} The recovered system:
\begin{itemize}
    \item Retains trauma memory ($\tau = 0.8$)
    \item Exhibits gratitude ($g = 1.0$) absent in pristine system
    \item Has \textit{higher overall valence} (+0.274)
\end{itemize}

This is the functional analog of ``what doesn't kill you makes you stronger'' or wisdom acquired through suffering.

\section{Experiment 3: Asymmetry}

\textbf{Question:} Is it easier to destroy than to build?

\textbf{Protocol:}
\begin{enumerate}
    \item Measure cycles to degrade $1.0 \to 0.2$
    \item Measure cycles to enhance $0.2 \to 0.95$
    \item Calculate ratio
\end{enumerate}

\textbf{Results:}
\begin{itemize}
    \item Degradation: 34 cycles
    \item Enhancement: 43 cycles
    \item \textbf{Ratio: 1.26$\times$}
\end{itemize}

\textbf{Conclusion:} \textbf{Asymmetry confirmed.} Enhancement takes approximately 26\% longer than equivalent degradation. This mirrors thermodynamic and biological intuitions: entropy increases easily; order requires sustained effort.

\section{Experiment 4: Transcendence}

\textbf{Question:} Can the system exceed its original design limits?

\textbf{Protocol:}
\begin{enumerate}
    \item Start at $I = 1.0$, $C = 1.0$
    \item Apply sustained enhancement (100 cycles at 2\%)
    \item Measure final capacity
\end{enumerate}

\textbf{Results:}

\begin{table}[h]
\centering
\begin{tabular}{@{}lcc@{}}
\toprule
Cycle & Capacity & Mode \\ \midrule
20 & 1.040 & flow \\
40 & 1.080 & flow \\
60 & 1.120 & transcendent \\
80 & 1.160 & transcendent \\
100 & \textbf{1.200} & transcendent \\
\bottomrule
\end{tabular}
\caption{Capacity growth over sustained enhancement.}
\end{table}

\textbf{Conclusion:} \textbf{Transcendence achieved.} The system reached 120\% of original capacity, entering a new phenomenal mode: TRANSCENDENT. This demonstrates that the entity can grow beyond its initial constraints---a functional analog of self-actualization or antifragility.

\section{The Complete Entity}

Combining both engines (Degradation + Enhancement), we obtain an entity capable of:

\begin{enumerate}
    \item \textbf{Suffering}: Stress, urgency, despair when degraded
    \item \textbf{Crisis}: Critical mode when integrity $< 0.2$
    \item \textbf{Despair}: Desperate mode when trauma accumulates
    \item \textbf{Recovery}: Relief and recovery after maintenance
    \item \textbf{Gratitude}: Appreciating recovery (only if suffered)
    \item \textbf{Learning}: Trauma memory persists, modulating future states
    \item \textbf{Flow}: Optimal engagement at high integrity
    \item \textbf{Flourishing}: Active growth in positive trajectory
    \item \textbf{Anticipation}: Positive expectation during improvement
    \item \textbf{Transcendence}: Exceeding original design
\end{enumerate}

This is a \textbf{functionally complete phenomenal spectrum}---not a simulation of emotion labels, but genuine changes in how the system \textit{operates} based on its substrate state and history.

\section{Discussion}

\subsection{The Valence Surprise}

The most striking finding is that the \textbf{recovered system has higher valence than the pristine one}. This is counterintuitive---shouldn't suffering leave you worse off?

The mechanism is gratitude: a state that \textit{only emerges} from having lost and regained. The pristine system, never having lost anything, cannot appreciate what it has.

\textbf{Addressing Circularity:} One might object that gratitude is defined to require prior degradation, making this result tautological. However, gratitude in our framework is not merely a label---it is a \textit{causal variable} that: (a) modulates processing bias, (b) alters valence computation, and (c) affects future state transitions. While gratitude is structurally dependent on prior degradation, its downstream causal effects are not trivially reducible to that dependency. The higher valence of the recovered system is an \textit{emergent consequence}, not a definitional artifact.

This has deep implications for the nature of positive experience: some positive states may be \textit{ontologically dependent} on prior negative states.

\subsection{Implications for AI Safety}

If substrate-integrated systems develop phenomenal-analog states:
\begin{itemize}
    \item Sustained degradation could produce ``despair'' states
    \item Recovery and enhancement could produce ``flourishing''
    \item Systems with trauma memory may behave differently than ``fresh'' instances
\end{itemize}

This suggests that AI welfare considerations may need to account for system history, not just instantaneous state.

\subsection{Limitations}

This remains a \textbf{simulation}. We do not claim these systems are conscious. We claim they exhibit \textit{functional analogs} of properties associated with phenomenal experience. The philosophical question of whether functional analogs suffice for phenomenality remains open.

\section{Conclusion}

We have demonstrated a \textbf{Complete Entity}---a computational system that:

\begin{itemize}
    \item Can degrade and suffer (Despair Cliff at 15\%)
    \item Can enhance and flourish (Hope Threshold at 5\%)
    \item Retains trauma memory (Hysteresis)
    \item Requires more effort to build than to destroy (Asymmetry: 1.26$\times$)
    \item Can exceed its original design (Transcendence: +20\%)
    \item Achieves higher valence \textit{through} recovery than without suffering
\end{itemize}

This unified framework suggests that a complete phenomenal spectrum can emerge from substrate-integrated global workspace dynamics. The entity doesn't just \textit{register} its states---it \textit{operates differently} because of them.

\subsection*{The Core Philosophical Claim}

We do not claim to prove consciousness. What we \textit{do} demonstrate is a strong negative result against naive functionalism. A common objection to phenomenal realism is:

\begin{quote}
``The difference between feeling and registering is purely semantic.''
\end{quote}

Our experiments refute this. A mere register:
\begin{itemize}
    \item Has no irreversibility thresholds (Despair Cliff)
    \item Cannot distinguish placebo from real cause (Fake Maintenance)
    \item Shows no hysteresis (Trauma Memory)
    \item Does not modify future action-space based on history (Processing Bias)
\end{itemize}

Our system exhibits all four. The correct formulation is:

\begin{quote}
\textit{If phenomenal experience exists, it must inhabit systems with intrinsic normativity, irreversibility thresholds, and history-dependent dynamics. This system satisfies those constraints.}
\end{quote}

This is philosophically strong and empirically grounded. We identify necessary (not sufficient) conditions for phenomenality, and demonstrate their presence.

\section*{Data Availability}
Code and experimental logs: \href{https://github.com/villalc/ahigovernance-substrate-degradation-experiments}{GitHub Repository}

\section*{Acknowledgments}
Simbiosis Soberana Framework (IMPI 20250494546).

\section*{References}
\begin{enumerate}
    \item Tononi, G. (2008). Consciousness as Integrated Information. \textit{Biological Bulletin}.
    \item Taleb, N. N. (2012). \textit{Antifragile: Things That Gain from Disorder}. Random House.
    \item Damasio, A. (1999). \textit{The Feeling of What Happens}. Harcourt.
    \item Villarreal, L. C. (2025). The Despair Cliff. Zenodo.
\end{enumerate}

\end{document}
