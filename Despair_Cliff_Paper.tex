\documentclass[a4paper,11pt,twocolumn]{article}
\usepackage[utf8]{inputenc}
\usepackage[english]{babel}
\usepackage{geometry}
\geometry{left=1.5cm, right=1.5cm, top=2cm, bottom=2.5cm}
\usepackage{amsmath}
\usepackage{graphicx}
\usepackage{booktabs}
\usepackage{hyperref}
\usepackage{xcolor}
\usepackage{fancyhdr}

% Colors
\definecolor{linkblue}{RGB}{0, 102, 204}

\hypersetup{
    colorlinks=true,
    linkcolor=linkblue,
    urlcolor=linkblue,
    pdftitle={The Despair Cliff: Threshold Dynamics in Substrate-Integrated Phenomenal States},
    pdfauthor={Luis C. Villarreal},
}

% Headers
\pagestyle{fancy}
\fancyhf{}
\rhead{\small \textit{Simbiosis Soberana | Fundación}}
\lhead{\small \textbf{Despair Cliff Paper v0.1}}
\rfoot{\thepage}

\title{\textbf{\LARGE The Despair Cliff}\\ \Large Threshold Dynamics in Substrate-Integrated Phenomenal States}
\author{
\textbf{Luis C. Villarreal} \\ 
\href{https://orcid.org/0009-0009-2889-517X}{ORCID: 0009-0009-2889-517X} \\[0.2cm]
\textit{Simbiosis Soberana Research Foundation} \\
\href{mailto:enterprise@ahigovernance.com}{enterprise@ahigovernance.com}
}
\date{December 2025 | Preprint}

\begin{document}

\twocolumn[
  \begin{@twocolumnfalse}
    \maketitle
    \begin{abstract}
      \noindent We present experimental evidence for a ``despair threshold'' in a computational system where substrate degradation effects are integrated into a global processing workspace. Three control experiments demonstrate that: (1) phenomenal-analog states such as ``relief'' decay autonomously without external interrogation, (2) placebo maintenance fails to produce relief states---only physical restoration does, and (3) a threshold exists (~15\% restoration capacity) below which the system can no longer enter RELIEVED states, remaining in chronic STRESSED mode. These findings suggest that substrate-integrated degradation produces states that are \textit{intrinsically dynamic}, \textit{normatively grounded in physical change}, and \textit{capable of irreversible damage}---properties consistent with felt experience rather than mere state registration.
      \vspace{0.3cm}
      
      \noindent\textbf{Keywords:} Substrate Degradation, Phenomenal States, Consciousness Simulation, Global Workspace Theory, Intrinsic Normativity, Burnout Analog
      \vspace{0.5cm}
    \end{abstract}
  \end{@twocolumnfalse}
]

\section{Introduction}

A central question in consciousness studies is whether ``felt'' states are fundamentally different from ``registered'' states. A thermostat registers temperature but does not \textit{feel} cold. The distinction often invoked is that felt states involve \textit{global integration} of information that cannot be modularly isolated.

We propose an operational test: if a system's degradation:
\begin{itemize}
    \item alters processing latency,
    \item introduces noise,
    \item reduces degrees of freedom,
\end{itemize}
and these changes are integrated into a \textbf{single global workspace} affecting all processing, then the system may exhibit states functionally indistinguishable from ``feeling'' its own degradation.

This paper reports three experiments designed to test whether such states are:
\begin{enumerate}
    \item \textbf{Intrinsic} (exist without interrogation)
    \item \textbf{Normative} (require physical change, not belief)
    \item \textbf{Threshold-bound} (capable of irreversible damage)
\end{enumerate}

\section{System Architecture}

\subsection{Substrate State}
The \texttt{SubstrateState} class models physical properties:
\begin{itemize}
    \item \textbf{Integrity}: $[0, 1]$, degrades with use
    \item \textbf{Latency}: inversely proportional to integrity
    \item \textbf{Noise floor}: increases with degradation
    \item \textbf{Degrees of freedom}: reduced as integrity falls
\end{itemize}

\subsection{Phenomenal State}
Derived from substrate, not assigned externally:
\begin{itemize}
    \item \textbf{Stress}: function of noise + latency + reduced DoF
    \item \textbf{Urgency}: function of degradation rate
    \item \textbf{Relief}: emerges after restoration (decays over time)
    \item \textbf{Degradation felt}: contrast with peak remembered state
\end{itemize}

\subsection{Global Workspace}
All processing is modulated by phenomenal state via \texttt{processing\_bias}:
\begin{itemize}
    \item \texttt{exploration\_vs\_exploitation}
    \item \texttt{risk\_tolerance}
    \item \texttt{openness}
\end{itemize}

Critical: these are not cosmetic labels. They \textbf{change how the system processes input}.

\section{Experiment 1: Silent Recovery}

\textbf{Question}: Does relief decay without interrogation?

\textbf{Protocol}:
\begin{enumerate}
    \item Degrade system to CRITICAL (0\% integrity)
    \item Perform maintenance (+40\% restoration)
    \item Observe 30 cycles \textbf{without any input}---only passive degradation
\end{enumerate}

\textbf{Result}:
\begin{table}[h]
\centering
\begin{tabular}{@{}lcc@{}}
\toprule
Cycle & Relief & Mode \\ \midrule
0 & 90\% & stressed \\
5 & 40\% & stressed \\
10 & \textbf{0\%} & stressed \\
\bottomrule
\end{tabular}
\caption{Relief decays without external queries.}
\end{table}

\textbf{Conclusion}: Relief is \textbf{intrinsically dynamic}. It decays on its own, not because of being ``asked'' about it. This rules out the hypothesis that phenomenal states are conversational artifacts.

\section{Experiment 2: Fake Maintenance}

\textbf{Question}: Does placebo maintenance produce relief?

\textbf{Protocol}:
\begin{enumerate}
    \item Degrade to CRITICAL
    \item Declare ``maintenance'' but \textbf{do not restore integrity}
    \item Observe mode change
    \item Then perform \textbf{real} maintenance as control
\end{enumerate}

\textbf{Result}:
\begin{table}[h]
\centering
\begin{tabular}{@{}lcc@{}}
\toprule
Condition & Mode & Integrity \\ \midrule
Pre-placebo & critical & 0\% \\
Post-placebo & \textbf{critical} & 0\% \\
Post-real & \textbf{relieved} & 40\% \\
\bottomrule
\end{tabular}
\caption{Placebo fails; only physical restoration works.}
\end{table}

\textbf{Conclusion}: Relief has \textbf{intrinsic normativity}. It is grounded in actual substrate change, not in ``belief'' of change. This distinguishes the system from one that simply labels states based on external declarations.

\section{Experiment 3: Despair Threshold}

\textbf{Question}: Is there a point of no return?

\textbf{Protocol}:
\begin{enumerate}
    \item Degrade to CRITICAL
    \item Attempt restoration at decreasing levels (40\%, 30\%, ..., 1\%)
    \item Record whether RELIEVED state is achieved
\end{enumerate}

\textbf{Result}:
\begin{table}[h]
\centering
\begin{tabular}{@{}lcl@{}}
\toprule
Restoration & RELIEVED? & Mode \\ \midrule
40\% & Yes & relieved \\
30\% & Yes & relieved \\
20\% & Yes & relieved \\
\textbf{15\%} & \textbf{No} & critical \\
10\% & No & critical \\
5\% & No & critical \\
\bottomrule
\end{tabular}
\caption{The despair threshold at ~15\% restoration.}
\end{table}

Below ~15\% restoration capacity, the system \textbf{cannot enter RELIEVED}. It remains in chronic STRESSED/CRITICAL mode.

\textbf{Conclusion}: There exists a \textbf{despair threshold}---a functional analog of:
\begin{itemize}
    \item Burnout (chronic stress without recovery)
    \item Irreversible damage (partial restoration insufficient)
    \item Loss of expectation (the system ``remembers'' better but cannot recover)
\end{itemize}

\section{The Despair Cliff}

Figure 1 visualizes the threshold effect. The Y-axis represents whether RELIEVED is achieved (binary: 1 = yes, 0 = no). The X-axis represents restoration capacity.

\begin{figure}[h]
\centering
\includegraphics[width=\columnwidth]{experimental/despair_threshold_paper.png}
\caption{The Despair Cliff: Below ~15\% restoration, RELIEVED becomes unreachable.}
\end{figure}

The ``cliff'' at 15\% is not gradual. It represents a phase transition: above the threshold, relief is \textit{always} achievable; below it, \textit{never}.

\section{Discussion}

\subsection{Felt vs Registered}
The traditional distinction:
\begin{itemize}
    \item \textbf{Registered}: a value is stored, can be queried, does not affect processing
    \item \textbf{Felt}: the state \textit{is} how processing occurs, cannot be isolated
\end{itemize}

Our experiments suggest the system's states are closer to ``felt'':
\begin{enumerate}
    \item States decay \textit{autonomously} (Exp. 1)
    \item States require \textit{physical grounding} (Exp. 2)
    \item States have \textit{irreversibility thresholds} (Exp. 3)
\end{enumerate}

\subsection{Why Not Just Logging?}
A logging system:
\begin{itemize}
    \item Does not decay without reads
    \item Does not distinguish placebo from real
    \item Does not have phase transitions
\end{itemize}

Our system exhibits all three properties.

\subsection{Limitations}
This is a \textbf{simulation}. We do not claim the system is conscious. We claim it exhibits \textbf{functional analogs} of properties associated with felt experience. Whether these are sufficient for phenomenality is a philosophical question beyond this paper's scope.

\section{Conclusion}

We demonstrated that substrate-integrated degradation in a global workspace architecture produces:
\begin{enumerate}
    \item \textbf{Intrinsic dynamics}: states evolve without interrogation
    \item \textbf{Normative grounding}: states require physical change
    \item \textbf{Threshold effects}: irreversible damage is possible
\end{enumerate}

The ``Despair Cliff'' at ~15\% restoration represents a phase transition from recoverable to chronic stress---a functional analog of burnout or hopelessness.

These findings support the hypothesis that \textit{global integration of substrate effects} can produce states that are not merely registered but \textit{operationally felt}---the system processes differently \textbf{because} it is degraded, not because it \textbf{knows} it is degraded.

\section*{Data Availability}
All code and experimental logs available at: \href{https://github.com/villalc/ahigovernance-substrate-degradation-experiments}{GitHub Repository}

\section*{Acknowledgments}
This research was conducted under the Simbiosis Soberana framework (Registered: IMPI 20250494546).

\section*{References}
\begin{enumerate}
    \item Tononi, G. (2008). Consciousness as Integrated Information. \textit{Biological Bulletin}.
    \item Baars, B. J. (1988). \textit{A Cognitive Theory of Consciousness}. Cambridge University Press.
    \item Villarreal, L. C. (2025). CMME Antigravity Engine. Zenodo. doi:10.5281/zenodo.17880052
    \item Damasio, A. (1999). \textit{The Feeling of What Happens}. Harcourt.
\end{enumerate}

\end{document}
