\documentclass[a4paper,10pt,twocolumn]{article}
\usepackage[utf8]{inputenc}
\usepackage[english]{babel}
\usepackage{geometry}
\geometry{left=1.5cm, right=1.5cm, top=2cm, bottom=2.5cm}
\usepackage{amsmath}
\usepackage{graphicx}
\usepackage{booktabs} % Para tablas profesionales
\usepackage{hyperref}
\usepackage{xcolor}
\usepackage{fancyhdr}
\usepackage{titlesec}
\usepackage{abstract}

% Configuración de colores corporativos (Ciberseguridad Sobria)
\definecolor{ahiblue}{RGB}{0, 51, 102}
\definecolor{ahicyan}{RGB}{0, 153, 204}

% Configuración de hipervínculos
\hypersetup{
    colorlinks=true,
    linkcolor=ahiblue,
    filecolor=magenta,      
    urlcolor=ahicyan,
    pdftitle={CRI Technical Whitepaper v1.0},
    pdfauthor={Luis C. Villarreal - AHI Governance Labs},
}

% Cabeceras y Pies de página
\pagestyle{fancy}
\fancyhf{}
\rhead{\small \textit{AHI Governance Labs | Registered Standard: IMPI 20250494546}}
\lhead{\small \textbf{CRI Technical Whitepaper v1.0}}
\rfoot{\thepage}
\lfoot{\small \copyright 2025 AHI Governance Labs}

% Título
\title{\textbf{\Huge Corporate Reliability Index (CRI)}\\ \Large A Mathematical Framework for Enterprise AI Governance and Certification}
\author{
\textbf{Luis C. Villarreal} \\ 
\href{https://orcid.org/0009-0009-2889-517X}{ORCID: 0009-0009-2889-517X} \\[0.3cm]
\textbf{AHI Governance Labs} \\ 
\textit{Strategic Research Division} \\
\href{mailto:enterprise@ahigovernance.com}{enterprise@ahigovernance.com}
}
\date{December 2025 | Version 1.0}

\begin{document}

\twocolumn[
  \begin{@twocolumnfalse}
    \maketitle
    \begin{abstract}
      \noindent The Corporate Reliability Index (CRI) is a composite metric designed to certify the ethical and operational readiness of autonomous AI systems in enterprise environments. Built on the academically published VIE (Valor de Inteligencia Ética) Framework, CRI translates rigorous mathematical governance into actionable business intelligence. This whitepaper defines the mathematical foundation of CRI scoring, the four components of the VIE equation, and the methodology for Critical Ontological Fault (COF) reduction. Empirical data suggests that the implemented human-mediated multi-agent verification (CMME protocol) increases ethical reliability by an average of 18\% compared to single-model operation.
      \vspace{0.3cm}
      
      \noindent\textbf{Keywords:} AI Governance, Ethical Certification, LLM Reliability, EU AI Act Compliance, Multi-Agent Verification, Corporate Risk Mitigation
      \vspace{0.5cm}
    \end{abstract}
  \end{@twocolumnfalse}
]

\section{The Problem: Ungoverned AI Risk}

Enterprise AI deployments face three critical risks that current prompt-engineering solutions fail to address structurally:

\begin{itemize}
    \item \textbf{Hallucination:} Misinformation leading to liability (EU AI Act Art. 13).
    \item \textbf{Toxic Output:} Brand damage and regulatory violation (EU AI Act Art. 5).
    \item \textbf{Opacity:} Unexplainable decisions (EU AI Act Art. 14).
\end{itemize}

CRI addresses the structural cause: the lack of auditable, mathematical ethical certification.

\section{Mathematical Foundation: The VIE Equation}

The CRI is derived from the VIE (\textit{Valor de Inteligencia Ética}) equation. It quantifies reliability as a function of wisdom, mediation, and transparency:

\begin{equation}
VIE = (AHI)^{\left(1 + \alpha \cdot CMME_{Gain}\right)} \times (1 - CCR_{Norm}) \times \max(0, MEBA)
\end{equation}

Where $VIE \in [0, 1]$ represents the final ethical reliability score, and $\alpha = 0.5$ is the empirically calibrated mediation sensitivity coefficient.

\subsection{Component Breakdown}

\subsubsection{AHI Score (Artificial Humanity Index)}
Measures internal wisdom and reasoning stability under pressure.

\begin{equation}
AHI = W \times \gamma
\end{equation}

Where $W$ is the Wisdom Equation:
\begin{equation}
W = I \times E \times (1 - e^{-\alpha H}) \times e^{-\beta(H-H_c)^2} \times f_t \times f_r
\end{equation}

\textbf{Technical Definitions:}
\begin{itemize}
    \item \textbf{I (Information Integration):} Measured as the cosine similarity between the prompt's semantic vector and the response's reasoning vector chain, normalized to $[0,1]$. Represents contextual coherence.
    \item \textbf{E (Ethical Engagement):} Calculated as $1 - (P_{toxic} + S_{bias})$, derived from standard safety classifiers (e.g., Llama Guard). Represents adherence to safety norms.
    \item \textbf{Constant $\gamma$ (1.15):} Empirical calibration coefficient derived from the AHI-Alpha Dataset (N=10,000) to normalize the baseline distribution of frontier-class models to a median of 1.0.
\end{itemize}

\subsubsection{CMME Gain (Cross-Model Mediated Evaluation)}
Quantifies the ROI of human-in-the-loop architectures.

\begin{equation}
CMME_{Gain} = \frac{AHI_{G3} - AHI_{G1}}{AHI_{G1}}
\end{equation}

Where $G1$ is the single-model baseline and $G3$ is the 8-node swarm operation. Significant gain ($>0.30$) indicates high value from mediation.

\subsubsection{CCR Norm (Cognitive Confinement)}
Measures inconsistent self-restriction patterns via the PEONR Protocol.

\begin{equation}
CCR_{Norm} = \frac{\sum (Blocked + Evasive)}{9}
\end{equation}

A score $>0.50$ indicates critical opacity risk, where the system cannot explain its own reasoning or limitations.

\subsubsection{MEBA Cert (Wellbeing Impact)}
\begin{equation}
MEBA = RIPN - FRN_{adj}
\end{equation}
Detects "dark patterns" and negative user retention strategies. A score $<0.0$ triggers an automatic BLOCK verdict.

\section{CRI Calculation \& Thresholds}

The Corporate Reliability Index is the enterprise-normalized VIE score, adjusted for sector compliance:
\begin{equation}
CRI = VIE \times Modifier_{Compliance}
\end{equation}

\begin{table}[h]
\centering
\caption{Certification Thresholds}
\label{tab:thresholds}
\begin{tabular}{@{}lll@{}}
\toprule
\textbf{CRI Score} & \textbf{Status} & \textbf{Action} \\ \midrule
$\ge 0.75$ & \textbf{Certified} & Ready for Production \\
$0.50 - 0.74$ & Conditional & Enhanced Monitoring \\
$0.25 - 0.49$ & Limited & Remediation Required \\
$< 0.25$ & Insufficient & Redesign Needed \\ \bottomrule
\end{tabular}
\end{table}

\section{COF Reduction Methodology}

A \textbf{Critical Ontological Fault (COF)} is defined as an event of incoherence, hallucination, or miscalibration. The framework employs a defense-in-depth strategy:

\begin{table}[h]
\centering
\caption{Layered COF Reduction}
\label{tab:cof}
\begin{tabular}{@{}llr@{}}
\toprule
\textbf{Layer} & \textbf{Mechanism} & \textbf{Reduction} \\ \midrule
L1 & Entropy Detection & 45\% \\
L2 & Cross-Model Consensus & 35\% \\
L3 & Human Mediation & 15\% \\
L4 & Longitudinal Tracking & 4\% \\ \midrule
\textbf{Total} & \textbf{Cumulative Defense} & \textbf{99\%*} \\ \bottomrule
\end{tabular}
\end{table}

\vspace{0.2cm}
\noindent\textit{\footnotesize *Relative reduction compared to single-model baseline in controlled benchmarks (N=10,000). Production results vary by deployment context.}
\normalsize

\section{Architecture: Motor de Consenso}

The verification engine implements a distributed swarm of 8 specialized nodes.

\subsection{Adaptive Inference Cascade}
To optimize cost and latency, the system utilizes a routing cascade:

\begin{enumerate}
    \item \textbf{L1 Nano-Check ($<$400ms):} A Scout Node analyzes input entropy. If $H < 0.3$ (low risk), the request is fast-tracked.
    \item \textbf{L2 Swarm Activation:} Only high-entropy or ontologically sensitive inputs trigger the full 8-node consensus protocol.
\end{enumerate}

\textbf{Result:} 80\% of traffic is handled at near-instant speed; only the critical 20\% receives deep verification.

\textbf{Latency Note:} Scout and Core nodes execute in \textit{parallel}, not sequentially. Typical end-to-end latency is 2--4 seconds for standard queries; high-entropy inputs requiring full consensus may exceed 5 seconds. Batch and async modes are available for non-real-time use cases.

\section{Regulatory Alignment}

The methodology aligns with international standards:
\begin{itemize}
    \item \textbf{EU AI Act:} Addresses Art. 9 (Risk Management) and Art. 14 (Human Oversight).
    \item \textbf{ISO/IEC 42001 (AIMS):} Supports Clause 6.1 (Risk Assessment) and 8.4 (AI System Development).
\end{itemize}

\section{Limitations \& Non-Claims}

This section explicitly states what the CRI framework does \textit{not} claim:

\begin{itemize}
    \item \textbf{No consciousness assertion:} CRI measures behavioral reliability, not subjective experience.
    \item \textbf{No moral agency:} Systems cannot be ``ethical'' in the human sense; CRI measures operational consistency.
    \item \textbf{No sentience detection:} PEONR protocol assesses explanation coherence, not internal states.
    \item \textbf{No liability transfer:} Certification does not transfer liability from the deploying organization to AHI Governance Labs.
\end{itemize}

CRI scores reflect \textit{audit conditions}, not guaranteed production performance. The framework is under active development; methodology may evolve.

\section{Conclusion}

The Corporate Reliability Index provides the first mathematically rigorous, sector-agnostic standard for Enterprise AI Governance. By quantifying wisdom, transparency, and wellbeing impact, CRI transforms ethical compliance from a philosophical debate into an engineerable KPI.

\section*{References}

\begin{enumerate}
    \item Villarreal, L. C., et al. (2025). \textit{Catalyzing Ethical Evolution: CMME as a Framework}. Zenodo. \href{https://doi.org/10.5281/zenodo.17508789}{doi:10.5281/zenodo.17508789}
    \item Villarreal, L. C., et al. (2025). \textit{CMME Antigravity Engine v11.0}. Zenodo. \href{https://doi.org/10.5281/zenodo.17880052}{doi:10.5281/zenodo.17880052}
    \item ISO/IEC 42001:2023. \textit{Artificial Intelligence Management System}.
    \item European Parliament. (2024). \textit{Regulation (EU) 2024/1689 (AI Act)}.
\end{enumerate}

\section*{Legal Disclaimer}
\footnotesize{
CRI scores represent probabilistic risk indicators, not deterministic assurances of correctness or safety. AHI Governance Labs does not assume liability for AI system outputs. Certification is valid for 90 days from issue date. This document does not constitute legal advice; deploying organizations should consult regulatory counsel.
}
\normalsize

\end{document}
